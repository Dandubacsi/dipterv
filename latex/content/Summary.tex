\chapter{Összefoglalás}

Az elkészített diplomatervem során elmélyedtem a Kelvin-szondás mérés részleteiben és kidolgoztam egy szimulációs tervet a Kelvin-szondás mérőfej elektrosztatikus paramétereinek számítására, valamint az ebből levezetett koncentrált paraméterű modell paraméterezésére. A szimulációs folyamatot jelentősen megkönnyítő automatizálás lehetőségét kihasználva, elmélyedtem a COMSOL Multiphysics által biztosított LiveLink környezetében és ennek eredményéül jelentős mértékben sikerült a szimulációkat meggyorsítani, valamint az utófeldolgozást elvégezni. A vizsgált konfigurációk elektrosztatikus paramétereinek meghatározása után lehetőség nyílt a konfigurációk között választani és a további tervezési feladatokat specifikálni.

Az elektrosztatikus szimulációk eredménye alapján a mérőfej mechanikai specifikációja is előállt. Ezen specifikációknak megfelelően megterveztem a Kelvin-szonda rezgő elektródáját meghajtó MEMS eszköz két különböző változatát melyek közelítően megfelelnek az általam támasztott specifikációknak. A megtervezett és méretezett eszközön végül egy termomechanikus szimulációval meggyőződtem annak működőképességéről. Elkészítettem egy numerikus szimulációt PYTHON programozási nyelvben, mely figyelembe veszi a mérőfej elektrosztatikus és termomechanikus karakterisztikáit a mérőfejen mérhető feszültségjel számítására. Felvázoltam egy lehetséges architektúrát a mérőfej jelének mérésére CMOS technológián figyelembe véve a termikus zajt. Megvizsgáltam a termikus beavatkozást biztosító ellenállások felhasználhatóságát a mérőfej kitérésének mérésére kihasználva a Piezorezisztív-hatást és elkészítettem az ellenállások megfelelő karakterisztikáit.

Az elektrosztatikus és termomechanikus szimulációk lefuttatását követően megvizsgáltam az általam tervezett mérőfej gyárthatóságát. A gyárthatóság vizsgálatánál egy SOI szeletet felhasználó nagyfrekvenciás áramkörök megvalósítására optimalizált technológiát használtam fel, melyet kiegészítettem a tokozást megelőző, utólagos lépésekkel, melyekkel az áramkör átalakítható egy MEMS eszközzé, a folyamat során megőrizve az eredeti áramköri kapcsolástechnikát a szelet felületén. Elkészítettem a mérőfej egyszerűsített változatához tartozó gyártástechnológiai lépéssort mellyel az egyszerűsített mérőfej a BME Félvezetőtechnológia Laboratóriumában legyártható az Energiatudományi Kutatóközpont Műszaki, Fizikai és Anyagtudományi Kutató Intézetének közreműködésével.
