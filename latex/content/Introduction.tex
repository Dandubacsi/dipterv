\chapter{Bevezető}

A mikroelektronikai méréstechnika és anyagtudomány egy igen fontos mérési módszere a Kelvin-szondás felületi potenciál mérés\cite{Kelvin_desc}, amely a gyártás- és vizsgálattechnológiai ágazatokban számos helyen fellelhető, így a módszer fejlesztése is fontos feladat. Egy lehetséges módszer, az áramkörök méretének csökkenésével analóg módon, a vizsgáló berendezések méretének csökkentése kihasználva a mikroelektromechanikai rendszerek (MEMS) által biztosított pontos és gyors működést. A mérőrendszer fejlesztése során egy újszerű meghajtási móddal kísérleteztem, a rezonanciafrekvencián történő mechanikai deformáció hőtáguláson alapuló megvalósításával.

MEMS-eknél nagy kitérések elérésére általában nagy méretű, meanderezett rugókat és beavatkozókat tartalmazó eszközöket használnak\cite{micromirror}\cite{large_out_of_plane}. Ezen eszközök hátránya azonban a relatíve nagy méret és az ezzel együtt járó viszonylag nagy tömeg, mely korlátozza a rendszer rezonanciafrekvenciáját és ezzel együtt a mérési sebességet. Egy másik út, a meanderes struktúrákkal szemben, a hosszú egyenes konzolokkal rendelkező struktúrák, melyek képesek jelentős kitéréseket produkálni\cite{brai}, ám a méretükből és tömegükből származó korlátok ezeket a struktúrákat is érintik. Dolgozatomban egy rövid karokkal és kis felülettel rendelkező MEMS struktúrát mutatok be, amely így mentesül az előbb említett hátrányoktól, azonban így a $\mu m$-es kilengések elérése nagyobb kihívás.

Diplomatervemben folytatom és továbbfejlesztem korábbi munkáim\cite{bsc}\cite{tdk2020} során szerzett tudást és ismereteket. Az itt bemutatott MEMS eszköz immár a harmadik változata annak a Kelvin-szondás méréshez készített mérőfejnek, melyet szeretnénk megvalósítani. Az első változatban\cite{bsc} egy torziós elven működő rugót és egy elektrosztatikus beavatkozót szerettem volna felhasználni a MEMS eszköz kitérítésére, azonban a tervezett konstrukció nem volt képes elegendően nagy kitéréseket produkálni, valamint nem a teljes működési tartományon volt stabil\cite{bsc_instability}. A második változathoz\cite{tdk2020} lecseréltem az elektrosztatikus meghajtást egy termomechanikusra, így elkerülve az elektrosztatikus meghajtásból adódó zavarokat. Ez a változat egy meanderes rugóval volt ellátva, valamint egy viszonylag nagy mérőfelülettel rendelkezett, így a működési frekvenciatartománya nem volt elegendően nagy.

A tervezett mérőfej bemutatása előtt azonban a munkám megértéséhez elengedhetetlen ismeretek átadásával kezdem a dolgozatomat, melyben először is áttekintem a Kelvin-szondás mérés módszertanát és ismertetem a szükséges fogalmakat és mennyiségeket. A MEMS mérőfej méretezéséhez egy elektromágneses és egy termomechanikus modellt állítok fel a geometriai paraméterek tervezéséhez. A diplomatervemben bemutatásra kerülnek a szükséges matematikai modellek és fogalmak, valamint a mérőfej számítógépes modelljei, a rajtuk végzett szimulációk és azok eredményei. A mérőfej megtervezése után kitérek a gyártástechnológiai megvalósíthatóság kérdésére is, valamint egy kezdetleges tesztstruktúrát is megadok, melyen a termomechanikus kitérést lehet tesztelni és mérni.
