\chapter*{Kivonat}\addcontentsline{toc}{chapter}{Kivonat}

Dolgozatomban bemutatok egy Kelvin-szondás mérésre alkalmas mikroelektromechanikai rendszer (MEMS) tervezési folyamatát. A Kelvin-mérés során a mintafelület elektrosztatikus potenciálviszonyait tudjuk vizsgálni, a mintával történő kontaktálás nélkül, így nem megzavarva annak felületi viszonyait. A Kelvin-mérés a félvezetőiparban egy elterjedt és jól ismert mérési metódus, mely remekül használható kutatási célokra.

A tervezés első fázisaiban a szükséges paraméterek megállapítása volt a cél, így elektromágneses szimulációkat végeztem, melyek segítségével specifikálható a tervezendő MEMS eszköz mechanikai szerkezete. Az elektromágneses szimulációkat a felületelem módszer felhasználásával végeztem el, így elkerülve a térfogati diszkretizációt és csökkentve a szükséges számítási kapacitás mértékét. A szimulálandó modelltér méretének meghatározásához egy analitikai modellt dolgoztam ki, mely a kialakuló elektromágneses mezőt megfelelően közelíti. A mező jellegének javítása érdekében egy fókuszálási módszert is hozzáadtam a mérőfejhez. A szimulációk eredményéül előálló elektromágneses tér alapján definiáltam a mérőfej érzékenységét és meghatároztam annak térbeli eloszlását. A szimulációk során előálló elosztott paraméterű modell alapján egy koncentrált paraméterű helyettesítőképet alkottam, melyet a későbbi számítások során fel is használtam. A Kelvin-szondás mérés kimeneti jeléül szolgáló jeleket a koncentrált paraméterű modell alapján közelítettem, mely a mechanikai specifikáció megalkotását tette lehetővé. A szimulációs és adatfeldolgozó munkafolyamatot MATLAB és PYTHON scriptek segítségével automatizáltam, kihasználva a COMSOL MULTIPHYSICS adta lehetőségeket. A szimulációk és utófeldolgozáshoz szükséges idő ezáltal jelentős mértékben csökkenthető volt.

Az elektromágneses térszámítás eredményeinek birtokában elkezdtem a mechanikai szerkezet méretezését és tervezését. A Kelvin-szondás méréshez szükséges rezgést termomechanikus csatolás lévén érem el, elkerülve az elektrosztatikus gerjesztés okozta interferenciát. A működési sebesség növelése érdekében a gerjesztést pár tíz-száz kHz frekvencián kell elvégezni, mely megfelelő méretezés mellett megegyezik a rendszer mechanikai rezonanciájával, ezáltal növelve a kitérés amplitúdóját. A szükséges számítási igény csökkentése érdekében tranziens szimulációt csak a mérőfej validációjára használtam, a méretezés során csak szinuszos állandósult állapotot vizsgáltam. Szükséges volt továbbá a termomechanikus, csatolt egyenletek szinuszos állandósult állapotbeli alakjának implementációja a szimulációs szoftverben. A mechanikai gerjesztést végző termikus beavatkozást optimalizáltam a nagyobb kitérések érdekében, valamint megvizsgáltam a lehetséges csillapítási mechanizmusokat.

A mérőfej tervezésének utolsó lépéseként megvizsgáltam a mérőfej gyárthatóságának feltételeit, annak CMOS kompatibilitását és elkészítettem egy kezdetleges tesztstruktúra tervét mellyel a termikus beavatkozó teljesítménye mérhető.

\chapter*{Abstract}\addcontentsline{toc}{chapter}{Abstract}

In my thesis I present the design process of a microelectromechanical system (MEMS) which is capable of performing a Kelvin probe measurement. During the measurement we probe the electrostatic potential of a sample without making conductive contact to it, therefore not disturbing its surface state. The Kelvin measurement is a widely known and accepted methodology in the semiconductor industry which is well suited for research purposes.

In the first phase of development my goals was to specify the necessary parameters, so first I simulated the electromagnetic (EM) field which made the specification of the mechanical structure of the MEMS device possible. I ran the EM simulations using the Boundary Element Method (BEM). Thanks to this, I avoided the need for three dimensional discretisation which in return reduced the required computing power. To estimate the size of the model space required to simulate I derived an analytical model, which approximated the resulting EM field. To optimize the EM field I added a focusing method to the probe's setup. From these simulations I defined the sensitivity of the probe and determined its spatial distribution. Using the distributed simulations I made a lumped element model to replace the original, which I used for further numerical calculations. The output signals of the Kelvin probe measurement was approximated using this lumped model, which made the specification of the mechanical model possible. The simulational and post processing workflow were automated by MATLAB and PYTHON scripts. These scripts utilized the built-in features of COMSOL MULTIPHYSICS that greatly reduced the time needed for the simulations and post processing.

Following the results of the EM field calculations I began designing the mechanical construction. The necessary vibration for the Kelvin probe is generated by a thermomechanical coupling in which way we avoid the interference caused by the usual electrostatic excitation. To increase the operating speed, the excitation is done in the tens to hundreds of kilohertz range, which in a well designed case is the same as one of the resonant frequencies of the probe therefore increasing the resulting amplitude of the vibration. To decrease the necessary computational power I only ran transient simulations to verify the probe, while for the design part I used a sinusoidal steady-state. To utilize the sinusoidal steady-state solver it was necessary to derive these versions of the coupled thermomechanical equations and to implement these in the software as well. The thermal excitation, which drives the mechanical system, was optimized for the greater amplitudes. I also examined the effect of a possible dissipation mechanism too.

The final stage of the design process was to examine the manufacturability, the CMOS compatibility of the designed Kelvin probe and to design a rudimentary test structure, which is capable of measuring the thermal actuator's performance.

\vfill
