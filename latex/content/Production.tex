\chapter{Gyártástechnológia koncepció}
\label{chap:manufacturing}

Az előző fejezetekben bemutatott szimulációs és méretezési lépések végeztével előállt a mérőfej végleges terve. A mérőfej gyártástechnológiai tervezése során az elsődleges szempont a Tanszéken található Félvezetőtechnológia Laboratóriumban történő stabil és megbízható gyárthatóság volt. Mivel CMOS gyártósor nem található a laboratóriumban, így a mérőfej áramköri részeit nem tudjuk saját magunk legyártani. Ezeket a részeket ki kell szervezni egy külső cégnek, melyek az áramköri tervek alapján le tudják gyártani az eszközt. Az áramkör legyártását követően a mérőfej egy chip formájában áll elő, hogy ebből a chip-ből előállítsuk a végleges MEMS eszközt, úgynevezett Post-CMOS technológiai lépéseket kell végezzünk a chipen.

A Tanszék számára elérhető gyártástechnológiák közül elsősorban a Circuits Multi-Projects\cite{CMP} (vagy röviden CMP) által kínált technológiákból válogattam. A megtervezett struktúra által támasztott követelményeknek megfelelően a gyártási eljárások közül a Silicon On Insulator (vagyis SOI) szeleteken megvalósított technológiákat létesítettem előnyben. Ezen eljárások közös tulajdonsága, hogy az aktív, áramköri elemeket megvalósító rétegek alatt egy szilícium-dioxidból álló eltemetett réteg található közel a szelet felületéhez. Ez az eltemetett réteg természetes marásgátló rétegként használható a szelet elvékonyítása során. A technológiák közül szintén előnyben részesítettem a vastag fémrétegeket alkalmazókat, ezzel megvalósítva a karok felületén a bimorf struktúrát.

A CMP által biztosított gyártástechnológiák közül a STMICROELECTRONICS 130 nm-es H9SOI-FEM\cite{H9SOI-FEM} eljárása tűnt a legjobb választásnak. A kiválasztott gyártástechnológia alapvetően rádiófrekvenciás áramkörök végpontjaihoz (RF Front End of Line) lett optimalizálva, azonban ez a lépéssor rendelkezett a számomra legkedvezőbb tulajdonságokkal. A gyártástechnológián előállíthatók a szokványos CMOS áramköri elemek, ilyenek a kis- és nagyfeszültségű N- és PMOS tranzisztorok, a sziliciddel adalékolt poliszilícium ellenállások, lineáris MIM (Metal Insulator Metal), MOM (Metal Oxid Metal) és nemlineáris MOS kapacitások, az ESD védelmet ellátó diódák valamint nagy jósági tényezővel rendelkező induktivitások. A gyártástechnológia által szolgáltatott tranzisztorok elsősorban a 2,5 V-os tartományra lettek méretezve, de található 1,2 V-os nagy sebességű és 8-13 V-os nagyfeszültségű tranzisztor is a technológiai könyvtárban. A gyártástechnológián használható sziliciddel adalékolt ellenállások négyzetes ellenállása 10 $\Omega / \square$ és 320 $\Omega / \square$, ezen értékek alapján állítottam be az általam számolt diffúziós ellenállások értékét is, hogy az ellenállások által foglalt területet is jól közelítse a modellem. A kiválasztott gyártástechnológia másik nagy előnye, hogy a rendelkezésre álló négy fémréteg közül a harmadik egy 4 $\mu m$ vastag réz, így a bimorf struktúra is megvalósítható az integrált áramkör gyártása során. Ennek a rétegnek a vastagsága alapján határoztam meg a modellben beállított rézréteg vastagságát.

Ezt a gyártástechnológiát felhasználva a MEMS eszköz áramköri része szokványos analóg integrált áramköri tervezéssel tervezhető. Ezt követően a megrendelt és legyártott chip-et különböző alacsony hőmérsékletű lépésekkel és marásokkal kialakítjuk a chipet körülvevő szilíciumból a MEMS eszköz számára fontos részeket. A pontos lépéssor meghatározásához szükségünk volt a gyártó által szolgáltatott dokumentációra, azonban ezeket még nem kaptuk meg, így a pontos gyártási lépéssor megtervezése nem volt lehetséges.

A végleges gyártási lépéssort nélkülözve inkább egy egyszerűsített változat elkészítését vázolom fel. Ennek az egyszerűsített változatnak a célja a mérőfej mozgatását megvalósító termomechanikus csatolás vizsgálata. Ennek következtében a kapacitív mérés pontossága csak másodlagos prioritást élvezett. Ezen gondolat mellett megtervezett mérőfej csak a diffúziós ellenállásokat és a réz réteget tartalmazza, mely a mérőfej egész felület fogja borítani. A Kelvin-szondás mérés így ugyan elvégezhető, azonban a rendszer felbontóképessége csak korlátozott lesz és nélkülözni fogja a fókuszáló elektróda hatását. Az egyszerűsített változat gyártási lépéseit igyekeztem a végleges chipet tartalmazó lépéssorhoz igazítani, azonban ez csak korlátozottan volt lehetséges a szükséges dokumentáció hiánya miatt. Az így kialakított gyártási lépéseket \aref{fig:manufacturing}. ábrán szemléltetem.

\Aref{fig:manufacturing}. ábrán a lépések jobb megértése érdekében két keresztmetszeti vetületet is ábrázolok. Ezek az AA' és BB' vetületek melyek a chip középvonala mentén függőlegesen és az egyik kart merőlegesen metszve haladnak. Kiindulásként a SOI szelet felülnézeti képét és a keresztmetszetet ábrázoltam (0. állapot).

Első lépésként a szelet felületén termikus oxidot növesztünk, mely az N diffúzió maszkolását fogja elvégezni. A növesztett oxidot fényérzékeny lakkal vonjuk be, majd ezt UV fénnyel levilágítjuk és kálium-hidroxidos oldatban előhívjuk. Az így előállított és maszkolt oxidot hidrogén-fluoridos oldatba mártjuk, mely eltávolítja a szelet felületén növesztett oxidot ahol azt nem védi a fényérzékeny lakk. Ezt követően eltávolítjuk a fényérzékeny lakkot tömény kálium-hidroxidos oldatba mártva a szeletet. Végül megtisztítjuk a szelet felületét a szerves maradványoktól. Az ebben a paragrafusban leírtakat a gyártás során többször fogjuk megismételni, így a későbbiekben csak litográfia lépésekként fogok erre hivatkozni.

A litográfia lépések után a szelet felületén kialakítjuk a termikus gerjesztést biztosító ellenállásokat egy kétlépcsős diffúziót használva (1. állapot). A diffúzió első lépéseként a szelet felületére leválasztjuk a megfelelő mennyiségű adalékatomot, majd a második lépésben a leválasztott atomokat behajtjuk a kristályrács mélyebb rétegeibe. A diffúzió pontos paramétereit \aref{tab:diffusion}. táblázatban megtalálhatjuk.

A következő lépésben a leírt litográfiai folyamattal kialakítjuk a következő maszkréteget. A szeletet ezután 80 $^\circ C$-s 22 $\%$-os (4 M) kálium-hidroxid oldatban maratjuk\cite{KOH_etch}, míg a karok melletti L alakú területeken el nem érünk az eltemetett oxidrétegig (2. állapot).

A szeletet ezután megtisztítjuk és teljes felületén termikus oxidot növesztünk. A növesztett oxidréteget elektromos szigetelőnek használjuk fel, hogy az erősen adalékolt diffúziós ellenállásokat elszigeteljük a felületi réz rétegtől. A bimorf struktúra kialakításához a mérőfej felületén fényérzékeny lakkal egy maszkréteget viszünk fel, melyen a szükséges területeken ablakokat nyitunk. Az így kialakított maszkon keresztül vákuumpárologtatással és galvanizálással 4 $\mu m$ vastag rézréteget növesztünk, majd eltávolítjuk a fényérzékeny lakkot (3. állapot).

Ezeket a lépéseket követően előáll az egyszerűsített mérőfej rétegszerkezete és csak a szilícium szelet többi részétől kell azt eltávolítani. Azért, hogy megvédjük a rétegszerkezetet a kálium-hidroxidos marástól, a szelet felületén CVD (Chemical Vapor Deposition) eljárással szilícium-nitridet viszünk fel (4. állapot). Ehhez a rétegfelvitelhez a tanszéki laboratóriumban nem áll rendelkezésre a szükséges eszközpark, így ezt a folyamatot az Energiatudományi Kutatóközpont Műszaki, Fizikai és Anyagtudományi Kutató Intézetének a segítségével végezzük el. Ennek a lépésnek egy alternatívája, mely általunk is elkészíthető, a spin-on oxid használata, melyet a fényérzékeny lakkhoz hasonlóan tudunk felvinni a felületre.

Az integrált áramköri gyártótól kapott tokozatlan chipünk rendelkezni fog az összes szükséges réteggel a chip vertikális keresztmetszete mentén, valamint egy passziválóréteggel, mellyel a felső rétegeket védik. Az  chip felületén litográfiával ablakokat nyitunk a kimarandó részeknek, ilyen a hátoldali vékonyítás valamint a 2. lépésnél kialakított karok elválasztása a tömbi szilíciumtól. Ezen ablaknyitások után a chip beültethető az eddig leírt gyártástechnológiába a 4. állapot után.

A szilícium-nitrid által nem védett területeket a már említett 22 $\%$-os oldattal lemarjuk az eltemetett oxidrétegig (5. állapot).

A mérőfej gyártásának utolsó lépése az eltemetett oxidréteg kémiai marása, melyet hidrogén-fluoridos oldattal tehetünk meg (6. állapot).

\imgsrc{figures/Production/manufacturing.png}{Az egyszerűsített mérőfej gyártási lépései}{fig:manufacturing}{0.65}
