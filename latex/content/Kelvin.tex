\chapter{Mérési eljárás}

\section{Kontakt potenciál kialakulása}
\label{contact}

A dolgozat témáját adó Kelvin-szondás mérés megértéséhez először szeretném tisztázni a mérés tárgyát adó fizikai jelenséget, a felületi kontakt potenciált és annak kialakulását. Két különböző anyagi minőséggel rendelkező szilárdtest fémes kapcsolatba hozásánál, az eltérő energetikai sávszerkezet miatt, a kontaktus felületén egy kiegyenlítődési folyamat indul meg, melyet a kilépési munkával jellemezhetünk. A szilárdtest kilépési munkája (angolul work function-je) alatt azt a termodinamikai energiamennyiséget értjük, mely egy a szilártesten belüli elektron eltávolításához szükséges.

\equref{dE = \mu\ dN}{eqn:chemical_pot}

\Aref{eqn:chemical_pot}. egyenletben E az energia, $\mu$ a kémiai potenciál és dN a részecskeszám változás. Ez az energia nem tartalmazza az elektron további erőterekből való eltávolítását - ilyen például az elektrosztatikus tér - így csak a kristályráccsal való kölcsönhatását szüntetjük meg. A kiszakított elektronok energiaszintjét definiáljuk vákuum szintnek és a további energiaszinteket ettől az energiaértéktől tudjuk mérni. Ezt felhasználva definiálhatjuk a kilépési munkát, ami a kémiai potenciál és vákuum szint energiakülönbsége.

\equref{W_f = E_\text{vákuum}-\mu}{}

A kémiai potenciál azzal az energiamennyiséggel egyezik meg, mellyel a szilárdtest összes termodinamikai energiája megváltozik, ha egy elektront hozzáadunk, vagy elveszünk a szilárdtestből, ahogy az \aref{eqn:chemical_pot}. egyenletből látszik is. A kilépési munka nem csak az anyagi minőség függvénye, hanem a tömbi anyag felületi viszonyaié is. Ilyenek lehetnek például a kötött töltések, lógó kötések, valamint a szennyező atomok jelenléte. Ez a kölcsönhatás teszi lehetővé, hogy a kilépési munka mérésével az anyag felületi viszonyairól is képet kapjunk.

A kontaktus határfelületén a kisebb kilépési munkával rendelkező, tehát átlagosan nagyobb potenciális energiájú elektronok, az energetikailag kedvezőbb alacsonyabb energiájú állapot felé áramlanak a diffúziónak köszönhetően. A termikus mozgás következtében az elektronok el tudnak jutni a kontaktus másik oldalára és ott be tudják tölteni az alacsonyabb energiaszinteket, így ezen a térrészen a nagyobb energiájú elektronok koncentrációja lecsökken. Ez a koncentrációkülönbség hajtja tovább a diffúziót a termikus mozgásnak köszönhetően. Ennek következtében a határfelület mentén egy töltött kettősréteg alakul ki. A kisebb kilépési munkájú anyagban a hiányzó elektronok lokalizált pozitív atomtörzseket hagynak hátra, míg a nagyobb kilépési munkájú térrészben lévő elektron többlet negatív tértöltésként jelentkezik. Ez a folyamat addig tart, míg a kialakuló töltés kettősréteg - vagy más szóval kiürített réteg - elektromos potenciálja nem kompenzálja a két térrész energiaszintjei közötti különbségét.\cite{mizsei} A folyamatot jól szemléltetik \aref{fig:contact_potential_1}. és \aref{fig:contact_potential_2}. ábrák, melyen két eltérő fém sávszerkezetét valamint az egymáshoz viszonyított kontakt potenciáljukat ábrázoltam. Az ábrán E = 0 a vákuum szintet jelöli, $e\phi_1$ és $e\phi_2$ a két fém kilépési munkáit, ahol e az elektromos töltés egysége és $\xi$ a kialakuló elektromos teret.

\imgsrclr{figures/Kelvin/initial_energy.png}{figures/Kelvin/equilibrium_energy.png}{Kiindulási energiaszerkezet}{Egyensúlyi energiaszerkezet}{fig:contact_potential_1}{fig:contact_potential_2}{1}{1}

A töltés kettősréteg térbeli változása leírható egy Poisson-Boltzmann-egyenlettel, hiszen a kilépési munkák és a töltéshordozók koncentrációja összefügg\cite{Garrett_Brattain}. Az egyenlet megoldásából jól látható\cite{poisson-boltzmann}, hogy a tértöltésréteg vastagsága a határfelülettől mérve függ az anyag dielektromos állandójától, valamint a mozgóképes töltéshordozók koncentrációjától. Fémek esetében a tértöltésréteg néhány atomsorig tart, így az ott kialakuló sávelhajlás elhanyagolható és a töltéssűrűség egy megfelelő Dirac-deltával közelíthető. Félvezetők esetében a kiürített réteg vastagsága elsősorban az adalékolástól függ és akár a $\mu$m-t is elérheti. Ennek következtében kis geometriai méretek esetén a felületi jellemzők nagymértékben befolyásolják a félvezető elektromos tulajdonságait.\cite{mizsei} Ez a folyamat felelős például a MOS tranzisztorok rövidcsatornás hatásainak egy részéért is, a Source és Drain elektródáknál lévő sávelhajlások egymásra lapolódásából fakadóan\cite{poisson-boltzmann}.

Azt a mélységet, amelyen belül a külső elektromos tér hatása még érzékelhető a szilárdtestek kategorizálására is használhatjuk. A fémek esetében a külső elektromos tér néhány atomi távolságig fejti ki hatását, így jó közelítéssel állíthatjuk, hogy az elektromos tér a fémekbe nem hatol be, és a fémek belsejében azonosan nulla. Szigetelők esetében viszont, az elektromos tér a szigetelő belsejében lévő dipólusokra forgatónyomatékot fejt ki, így polarizálva azokat. Ennek következtében a szigetelő belsejében egy elektromos tér alakul ki, vagyis a külső tér a szigetelő teljes térfogatában kifejti hatását. A fémekkel és a szigetelőkkel ellentétben a félvezetőknél a külső elektromos tér hatása az előbb említett két eset között helyezkedik el, vagyis az adalékolástól függően változik a külső elektromos tér befolyásának mértéke. Ugyanakkor a geometriai méretek csökkenésével a félvezetők egyre inkább szigetelőként kezdenek viselkedni, a fogalom ezen értelmében, vagyis a külső elektromos tér hatását nem árnyékolják le.

Az eddigi ábrákon lévő energiaskála az elektronok termodinamikai energiáit tartalmazta, ugyanakkor kihagyta az elektrosztatikus kölcsönhatásból származó potenciális energiát. A kiegyenlítődési folyamatok jobb megértése érdekében a sávdiagramokon az energiákhoz hozzáadjuk az elektrosztatikus potenciálból származó járulékokat is. Ezzel a járulékkal kiegészített energiaskálát elektrokémiai potenciálnak nevezzük. Ennek a potenciálnak a használatával láthatóvá válik a termodinamikai egyensúly kialakulása, hiszen a két anyagban az elektrokémiai potenciálok azonos energiaszintre kerülnek termikus egyensúlyban. \ref{fig:electrochemical_potential}. ábra. Az elektrokémiai potenciál alkalmazásának hátránya, hogy a vákuum szint értéke nem lesz homogén, ahogy az az ábrán is látható.

\imgsrc{figures/Kelvin/electrochemical_potential.png}{Egyensúlyi állapot elektrokémiai potenciállal ábrázolva}{fig:electrochemical_potential}{0.5}

\equref{\phi_{kontakt} = \phi_1 - \phi_2 = \frac{W_{ki1}-W_{ki2}}{e}}{}

Az egyenletben $\phi_1$ és $\phi_2$ a két térrész mellett mérhető elektrosztatikus potenciál, $W_{ki1} W_{ki2}$ az anyagok kilépési munkái és e az elektromos töltés egysége. Az ábrán jól látható, hogy a kontaktus határfelülete mentén egy elektrosztatikus potenciálugrás alakul ki, ez az úgynevezett kontakt potenciál, melynek értéke a két fém kilépési munkáinak különbségéből fejezhető ki. Ennek a potenciálnak a felületi feltérképezésével láthatóvá válnak a vizsgált felületen a különböző adalékolású területek, valamint a tömbi anyag felületi állapotainak hatása is.\cite{surface_states} Ha ismerjük a felület adalékolásának eloszlását, valamint a felületi állapotok lokalizált hatásait akkor készíthetünk ezekből egy felületi eloszlástérképet, mellyel az adott felületi réteg minősítését végezhetjük el. Ilyen minősítő jelzők lehetnek az adaléksűrűség eloszlás, a maszkolás pontossága vagy a felületi szennyeződések darabszáma. Ebből is látszik, hogy a felületi potenciál feltérképezése rendkívül hasznos információt tud szolgálni.

A kontakt potenciálok létrejötte azonban azok mérését is nehezíti, hiszen ez a potenciál nem mérhető szokványos fémes kontaktusokon alapuló potenciálmérési módszerekkel. A jelenség oka, hogy a fémes kontaktus létrejöttekor a kialakuló potenciálugrások a mérőműszer és a minta felülete között pontosan kiegyenlítik a mérendő potenciált, így a teljes rendszer termodinamikai egyensúlyba kerül. Amíg ez az egyensúly nem áll be, addig a szilárdtesten belül létezik egy energiakülönbség, mely tovább tudja hajtani a diffúziót a kiegyenlítődés állapota felé.

\section{Kelvin-szondás mérés}

Az elektrosztatikus potenciál mérésére kézenfekvő egy kapacitív csatolás alkalmazása, ahol a minta felületéhez közel elhelyezett mérő elektróda és a minta felülete között az elektrosztatikus tér teremt kapcsolatot. A mérőfej és a minta közötti csatoló kapacitás pontos ismeretében elviekben számítható lenne a vizsgált felületi potenciál pontos értéke, azonban a mérési szórások ezt a mérési elvet igencsak megbízhatatlanná teszik. A vizsgálati módszer pontosabbá tételére egy lehetséges megoldás a mérést befolyásoló mennyiségek időbeli változtatása, valamint egy közvetett mennyiség mérése, mely a változó paraméterek mellett is konstans tud lenni kiegyenlített esetben. A gyakorlatban ez az időbeli változás a mérőfej-minta távolságának változtatása, melynek következtében a csatoló kapacitás értéke is változik. Az időben változó kapacitás következtében, állandó feszültség mellett, a kondenzátor töltése változik az idő függvényében, ezeket a töltéseket a kondenzátort töltő áram biztosítja. Ha a mérőelektróda potenciálját a vizsgált felület potenciáljával azonosra állítjuk be, úgy a csatoló kondenzátor kapcsain 0 V feszültség keletkezik, így a változó kapacitás mellett is nulla töltés halmozódik fel az elektródán, tehát a töltőáram is nullává válik. Ezen nullaátmenet detektálása teszi lehetővé a felületi potenciál érintésmentes mérését.

A mérőelektróda mintától vett távolságát tetszőleges módon változtathatjuk, azonban ha a távolságot egy adott frekvenciájú periodikus jel szerint változtatjuk, úgy a kialakuló áramjel csak az adott frekvenciát és annak felharmonikusait fogja tartalmazni az időfüggő kapacitás miatt. A legkézenfekvőbb periodikus gerjesztés a harmonikus függvények használata. Mivel a gerjesztés frekvenciája a mérés során ismert, így a mérendő jelben ismertek a mérésből származó információt hordozó frekvenciakomponensek, így ezekre a frekvenciákra jobban illesztett feldolgozó áramkörök és jelfeldolgozás készíthető. A mérési eljárás sematikus felépítését  \aref{fig:measurement_sketch}. ábrán látható.

\imgsrc{figures/Kelvin/measurement_sketch.png}{Kelvin-szondás mérés sematikus összeállítása.}{fig:measurement_sketch}{0.5}

Az ábrán a minta felületi potenciálját az ismeretlen  $\phi_{\text{mérő}}$ értékű feszültségforrás reprezentálja, míg a mérendő potenciált a $\phi_{ismeretlen}$ értékű feszültségforrás. A mérés során a mérőelektróda és a mérendő felület közötti távolságot szinuszosan változtatjuk és mérjük az áramkörben folyó áramot. Az adott elrendezésben ennek következtében a változó kapacitású kondenzátor árama is változik. Az áram mérésére annak megzavarása nélkül csak szűkös elektronikai megvalósítások állnak rendelkezésre, hiszen itt egy átfolyó áramot kell mérnünk a potenciálok megváltoztatása nélkül. Erre egy lehetőség integrált áramköri technológiákkal a mágneses csatoláson alapuló mérés\cite{CMOS_I_measure}, azonban az mérés érzékenysége ($1\mu A/A$) az általunk használt töltőáramokkal ($nA-\mu A$) mérhetetlenül kicsin áramokat eredményezne. Ha eltekintünk a potenciálok meg nem változtatásától, akkor az átfolyó áramokat egy sönt ellenállással át tudjuk konvertálni feszültségé. Ezen feszültségek könnyen mérhetőek számos kapcsolástechnikával a söntön átfolyó áram megzavarása nélkül. Az így beiktatott sönt azonban szétválasztja a mérőelektróda potenciálját a táphoz képest, így a mérési kapcsolást ki kell egészíteni egy potenciál ($\phi$) méréssel is. A sönt ellenállás miatt a mérőfej érzékenyebb lesz a parazita kapacitásokra is, így a vizsgálatok során azokat is figyelembe kell venni.
